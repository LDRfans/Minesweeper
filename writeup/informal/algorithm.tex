\documentclass{article}

% if you need to pass options to natbib, use, e.g.:
%     \PassOptionsToPackage{numbers, compress}{natbib}
% before loading neurips_2020

% ready for submission
% \usepackage{neurips_2020}

% to compile a preprint version, e.g., for submission to arXiv, add add the
% [preprint] option:
%     \usepackage[preprint]{neurips_2020}

% to compile a camera-ready version, add the [final] option, e.g.:
%     \usepackage[final]{neurips_2020}

% to avoid loading the natbib package, add option nonatbib:
     \usepackage[nonatbib]{neurips_2020}

\usepackage[utf8]{inputenc} % allow utf-8 input
\usepackage[T1]{fontenc}    % use 8-bit T1 fonts
\usepackage{hyperref}       % hyperlinks
\usepackage{url}            % simple URL typesetting
\usepackage{booktabs}       % professional-quality tables
\usepackage{amsfonts}       % blackboard math symbols
\usepackage{nicefrac}       % compact symbols for 1/2, etc.
\usepackage{microtype}      % microtypography

\usepackage[ruled]{algorithm2e}
\usepackage{amsmath}

\title{Algorithms (Informally)}

% The \author macro works with any number of authors. There are two commands
% used to separate the names and addresses of multiple authors: \And and \AND.
%
% Using \And between authors leaves it to LaTeX to determine where to break the
% lines. Using \AND forces a line break at that point. So, if LaTeX puts 3 of 4
% authors names on the first line, and the last on the second line, try using
% \AND instead of \And before the third author name.

% \author{%
%   David S.~Hippocampus\thanks{Use footnote for providing further information
%     about author (webpage, alternative address)---\emph{not} for acknowledging
%     funding agencies.} \\
%   Department of Computer Science\\
%   Cranberry-Lemon University\\
%   Pittsburgh, PA 15213 \\
%   \texttt{hippo@cs.cranberry-lemon.edu} \\
  % examples of more authors
  % \And
  % Coauthor \\
  % Affiliation \\
  % Address \\
  % \texttt{email} \\
  % \AND
  % Coauthor \\
  % Affiliation \\
  % Address \\
  % \texttt{email} \\
  % \And
  % Coauthor \\
  % Affiliation \\
  % Address \\
  % \texttt{email} \\
  % \And
  % Coauthor \\
  % Affiliation \\
  % Address \\
  % \texttt{email} \\
% }

\begin{document}

\maketitle

\begin{abstract}
  Some infomal algorithms.
\end{abstract}

\section{Linear model}

\begin{algorithm}[H]
\caption{linear model}
\LinesNumbered
\KwIn{$X^T = [x_1,x_2,\cdots,x_8]$ (eight neighbors),
with $x_i = \{-1,1,2,\cdots,8\}$(-1 for unrevealed tiles).}
\KwOut{category $G = \{0,1\}.$(0 for safe and 1 for a mine)} \vspace{0.1in}
Assume $X \leftarrow \begin{bmatrix} 1 \\ X \end{bmatrix}$, $w^T = [w_0,w_1,\cdots,w_8].$\\
Begin by probing a corner square (Assume $(0,0)$)\;
\While{not game over}{
  Array \leftarrow tiles at frontier\\
  \For{tile in Array}{
    G(tile) \leftarrow $sign(X(tile)^Tw)$
  }
}

\end{algorithm}



\section{Dummy Q-learning}

\begin{algorithm}[H]
  \caption{dummy q-learning}
  \LinesNumbered
  \KwIn{$X^T = [x_1,x_2,\cdots,x_8]$ (eight neighbors),
  with $x_i = \{-1,1,2,\cdots,8\}$(-1 for unrevealed tiles).}
  \KwOut{category $G = \{0,1\}.$(0 for safe and 1 for a mine)} \vspace{0.1in}
  Assume $X \leftarrow \begin{bmatrix} 1 \\ X \end{bmatrix}$, $w^T = [w_0,w_1,\cdots,w_8].$
  \end{algorithm}


\end{document}